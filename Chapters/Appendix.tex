\section{About Meshes in COMSOL Multiphysics}.

In the course of developing my thesis, particularly while working on the modeling of infrared light through a glass substrate using COMSOL Multiphysics \texttrademark's Wave Optics module, I encountered a significant challenge related to mesh generation. The initial attempt to create a mesh for the silica glass model, a slab with a radius of one meter and a thickness of 5 mm, resulted in an error message from COMSOL. The error indicated that the current physics settings necessitated a mesh comprising approximately 11 million elements, a requirement that raised concerns about potential out-of-memory errors or process locking. This requirement stemmed from the model's attempt to simulate the behavior of infrared light with a 1-micron wavelength within the specified computational volume.

The root cause of the issue as the physics-controlled meshing process attempting to generate a mesh that resolved details down to the wavelength of the light being modeled. Two potential solutions were proposed: significantly reducing the computational volume of the model or altering the computational method to one that did not necessitate resolution at the wavelength scale. Opting for the latter approach, I adjusted the computational method accordingly, which effectively addressed the meshing issue and allowed the project to proceed without the need for an excessively dense mesh.

When selecting a mesh size, it's crucial to take into account the scale at which your model operates. If your geometry is measured in meters but you're examining phenomena occurring at the nanometer wavelength scale, choosing an excessively fine mesh could lead to huge computational demands, including significant memory usage. To explore the physics of your model efficiently and within a reasonable timeframe, a coarser mesh may be more appropriate. This approach balances the need for detail with practical considerations of computing resources.