The initial chapters laid the groundwork by illustrating the critical role of cooling technologies in addressing the pressing issues of global warming and the energy crisis. With an emphasis on PDRCs, we embarked on a journey through the theoretical underpinnings and practical applications of these devices, which offer the potential for efficient, energy-free cooling by radiating heat directly to outer space during daylight hours.

As we culminate this research, this concluding chapter aims to distill the insights gained from our investigations, outlining the contributions of this work to the field of cooling technologies and the broader context of sustainable energy solutions.

\section{My work}

COMSOL Multiphysics \texttrademark \space has proven itself as an indispensable tool for computational modeling, facilitating the simulation of intricate PDRC structures with relative ease. The modeling approach, exemplified in Chapter 3 through a detailed walkthrough with a busbar example, laid the foundation for all subsequent simulations.

This thesis has not only validated my COMSOL models against established theory but has also ventured into the modeling of Fresnel equations to observe the behavior of reflection coefficients across varying angles of incidence.

The core of our efforts was the computational modeling of PDRC designs from Hudgings Lab, layer by layer, revealing critical insights into their optical performance. Notably, the simulation results confirmed the imperative for PDRCs to exhibit high reflectivity within the solar spectrum, a requirement clearly illustrated by the limited reflectivity observed in the sole glass substrate model.

\section{Future Directions}

Moving forward, several avenues for further research can be identified:

\begin{enumerate}
    \item Enhanced definition in the reflectance versus wavelength graphs for glass plus silver and glass plus silver plus PDMS models, possibly through the utilization of more detailed refractive index interpolation tables or functions.
    \item Exploration of adding more materials on top of the PDMS layer to fulfill PDRC design criteria more effectively, experimenting with various thicknesses and refractive indices to exploit constructive interference across multiple interfaces more efficiently.
    \item The translation of computational models into physical prototypes for empirical validation in lab settings, comparing simulated reflectance with real-world performance.
    \item Development of reflectance versus angle of incidence models to aid in the characterization of PDRC behavior throughout the day, providing a benchmark for experimental testing on rooftop installations.
\end{enumerate}

Building on these initiatives, we see a profound link between the principles of Distributed Bragg Reflectors (DBRs) and Passive Daytime Radiative Cooling Systems (PDRCs). PDRCs strive to facilitate cooling without the need for active energy input by effectively radiating heat into the expansive coldness of outer space. The spectral selectivity and reflective capabilities intrinsic to DBR structures could vastly improve PDRCs.

By emulating the multilayer approach integral to DBRs, PDRCs can fine-tune their design to reflect a substantial portion of the solar spectrum, thereby reducing heat gain, while promoting the emission of thermal infrared radiation to enhance cooling. This alignment of high solar reflectance and high thermal emittance is the cornerstone of PDRC technology, presenting a pathway to energy-efficient cooling solutions.

\section{Final Remarks}

This thesis not only validates the utility of PDRCs within the context of sustainable cooling technologies but also charts a course for their further development and application especially on the computational modeling front.

As we look to the future, the insights and methodologies presented here will serve as a valuable resource for those endeavoring to advance the efficiency and applicability of PDRCs and similar technologies in our collective pursuit of environmental sustainability and energy efficiency.