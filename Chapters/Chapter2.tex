A Passive Daytime Radiative Cooling (PDRC) device operates by absorbing a lower amount of blackbody radiation than it emits, thereby facilitating electricity-free cooling, even in daylight conditions. Consequently, one of the pivotal attributes of a PDRC device is the imperative for an absorptivity ($\alpha$) as close to 0\% or, conversely, a reflectivity ($R$) of 100\% within the solar spectrum (ranging from 0.3 to 2.5 micrometers). This specification ensures that the device's surface remains entirely unaffected by solar heating during daylight hours.

To enhance the effectiveness of PDRC, it becomes essential to accurately measure and optimize this reflectivity ($R$) within the solar spectrum. One approach for achieving this goal is to perceive light as an electromagnetic wave, and from this perspective, derive a quantifiable means to measure $R$ through the renowned \textit{Fresnel equations}.

The Fresnel equations are mathematical expressions that delineate the proportion of incident energy that is either transmitted or reflected at the interface of two materials with differing refractive indices. This concept aligns precisely with our objectives, as we plan to stack plane surfaces featuring distinct reflective properties and refractive indices. This chapter serves as an exploration of the theoretical framework underpinning the derivation of $R$ via the Fresnel Equations, delving into associated phenomena such as total internal reflection. Additionally, we explore the practical application of these principles to PDRC devices.



\section{Fresnel Equations}
