Embarking on the findings of this chapter, we transition from the established COMSOL modeling workflow to the critical stage of validating theoretical concepts through the lens of computational analysis. The essence of this chapter is twofold: first, to harness the computational prowess of COMSOL in substantiating key optical phenomena such as anti-reflectivity thus corroborate our simulation outcomes with theoretical models; and second, to attempt to model the simplest Passive Daytime Radiative Cooling devices (PDRCs) at Hudgings Lab by showing how their reflectance varies with wavelength.

Our investigation begins with an empirical analysis of anti-reflective behaviors, examining both simple and composite multilayer arrangements. These simulations are meticulously compared to the theoretical predictions presented in \emph{Introduction to Optics} by Frank L. Pedrotti et al., ensuring our findings adhere to the rigorous standards of optical precision.

We delve into the intricacies of high reflectance by simulating stacks of alternating high and low refractive index layers, shedding light on the nuanced ways layer configurations impact reflectance. These insights are bench-marked against scholarly literature, reinforcing the accuracy of our models.

Aligning with practical applications, we employ the Fresnel equations to map out the reflection coefficient, for both the transverse electric (TE) and transverse magnetic (TM) polarization, against the angle of incidence. The insights gleaned here will serve to consolidate our understanding of various optical phenomena.

Moreover, the thesis explores the stratified design of Passive Daytime Radiative Cooling (PDRC) devices. By systematically layering glass, silver, and Polydimethylsiloxane (PDMS), we dissect the interplay of thermal and optical properties crucial to the development of passive cooling systems.

This chapter stands at the intersection of physics and simulation, merging the realms of theory and experimentation. It lays the foundation for future inquiries, providing the trajectory for subsequent research that leverages our detailed simulations. The collection of results, graphs, and tables here encapsulate more than just data; they are catalysts for continuous inquiry, propelling the quest for deeper understanding in the field of optical physics.

\section{COMSOL: Modeling Anti-reflectivity.}