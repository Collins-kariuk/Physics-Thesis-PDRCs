Embarking on the findings of this chapter, we transition from the established COMSOL modeling workflow to the critical stage of validating theoretical concepts through the lens of computational analysis. The essence of this chapter is twofold: first, to harness the computational prowess of COMSOL in substantiating key optical phenomena such as anti-reflectivity thus corroborate our simulation outcomes with theoretical models; and second, to attempt to model the simplest Passive Daytime Radiative Cooling devices (PDRCs) at Hudgings Lab by showing how their reflectance varies with wavelength.

Our investigation begins with an empirical analysis of anti-reflective behaviors, examining both simple and composite multilayer arrangements. These simulations are meticulously compared to the theoretical predictions presented in \emph{Introduction to Optics} by Frank L. Pedrotti et al., ensuring our findings adhere to the rigorous standards of optical precision.

We delve into the intricacies of high reflectance by simulating stacks of alternating high and low refractive index layers, shedding light on the nuanced ways layer configurations impact reflectance. These insights are bench-marked against scholarly literature, reinforcing the accuracy of our models.

Aligning with practical applications, we employ the Fresnel equations to map out the reflection coefficient, for both the transverse electric (TE) and transverse magnetic (TM) polarization, against the angle of incidence. The insights gleaned here will serve to consolidate our understanding of various optical phenomena.

Moreover, the thesis explores the layered design of PDRCs. By systematically layering glass, silver, and Polydimethylsiloxane (PDMS), we dissect the optical properties crucial to the development of passive cooling systems.

This chapter stands at the intersection of physics and simulation, merging the realms of theory and experimentation. The collection of results and graphs here encapsulate more than just data; they are catalysts for continuous inquiry and deeper understanding of optical physics as it related to PDRCs.

\section{COMSOL: Modeling Anti-reflectivity.}
Recalling the formula for the reflectance of a two-layer anti-reflecting film, given by equation \ref{reflectance for 2-layer antireflecting films}, and assuming light strikes normally, we have:

\begin{equation}\label{reflectance for 2-layer antireflecting films - chap4}
    R = \left(\frac{n_0n_2^2 - n_sn_1^2}{n_0n_2^2 + n_sn_1^2}\right)^2
\end{equation}

Here, we anticipate zero reflectance when the expression in the numerator becomes zero, leading us to equation \ref{zero reflectance criterion}, which stipulates the condition for achieving no reflectance:

\begin{equation}\label{zero reflectance criterion - chap4}
    \frac{n_2}{n_1} = \sqrt{\frac{n_s}{n_0}}
\end{equation}

This relationship delineates the necessary ratio between the refractive indices of the layers that leads to a reflectance of zero.

When working with a glass substrate of refractive index $n_s = 1.5$ and air of refractive index $n_0 = 1$, the ideal refractive index ratio, $\frac{n_2}{n_1}$, is calculated to be $\sqrt{1.5} \approx 1.225$.

In COMSOL, our goal is to identify and define (in our parameters table) two materials for the double-layer structure that closely match this ratio to maximize anti-reflective properties. It's crucial to note that within the visible spectrum, ranging from 400 to 700 nm, each wavelength uniquely interacts with materials based on their specific refractive index at that wavelength. This variability, known as dispersion, means that optimal thickness and refractive indices for minimizing reflectance at one wavelength may not be as effective at another. While it's challenging to select materials that precisely align their refractive indices to achieve exactly zero reflectance, our aim is to get as close to 0\% reflectance as practically possible.

The limitation of a quarter-wavelength layer is its selective reflectivity; it effectively eliminates reflection at a specific wavelength but substantially reflects at other wavelengths in proximity. Additionally, its performance varies significantly with the angle at which light strikes the surface. A viable solution to overcome this is the application of multilayer coatings. These coatings, compared to their single-layer counterparts, tend to lower the reflection coefficient over a broader spectrum and accommodate a greater diversity of tangible materials \cite{pedrotti_introduction_2007}.

In the following section, we will compare the reflectivity of two distinct multilayer coatings across an extensive spectral range. The analysis will cover a dual-layer, quarter-quarter wavelength coating, and a triple-layer, quarter-half-quarter wavelength coating. It will be demonstrated that the triple-layer coating achieves a more uniform low reflectance throughout the majority of the visible light spectrum.


% TODO: CHANGE the title of this subsection

\subsection{Anti-reflectance Layers: Setup}

The pursuit of anti-reflective coatings leads us to the utilization of thin film dielectrics within COMSOL. In line with the criteria for zero reflectance, the first dielectric film, starting from the base, is cerium trifluoride (CeF3) with a refractive index of 1.63, while the second film is magnesium fluoride (MgF2) with a refractive index of 1.38. The refractive index ratio of approximately 1.181 aligns closely with the optimal ratio of 1.225. As established in the previous chapter, effective anti-reflectivity for a dual-layer arrangement requires each film to be a quarter-wavelength in thickness, which corresponds to the wavelength at which minimal reflectance is desired.

In our enhanced three-layer structure, designed to broaden anti-reflectivity across a larger wavelength spectrum, the middle layer is a half-wavelength film of zirconium dioxide (ZrO2), chosen for its refractive index of 2.1. This multi-layer configuration allows for increased anti-reflective performance compared to the more wavelength-specific two-layer model.

This screenshot from the COMSOL desktop highlights the configuration for modeling anti-reflective coatings. Pay particular attention to the 'Film Properties' area within the 'Thin Dielectric Film' settings where the quarter-wavelength calculation is outlined.

\begin{figure}[ht!]
  \centering
  \includegraphics[width=0.4\textwidth]{Chapters/Figures/Chapter 4 Figures/COMSOL Desktop Showcasing Antireflectivity Setup.png}
  \caption{COMSOL desktop setup showcasing anti-reflectivity modeling.}
  \label{fig:COMSOL desktop showcasing antireflectivity}
\end{figure}

Here, the findings from the anti-reflectivity modeling using CeF3 and MgF2 for the quarter-quarter wavelength combination, as well as CeF3, MgF2, and ZrO2 for the quarter-half-quarter wavelength setup, are presented.

\begin{figure}[ht!]
  \centering
  \includegraphics[width=0.4\textwidth]{Chapters/Figures/Chapter 4 Figures/Quarter-Half-Quarter.png}
  \caption{Illustration of the reflectance behavior for two anti-reflective coating configurations across the visible light spectrum. The blue line indicates the performance of a quarter-quarter wavelength coating, whereas the green line shows the effect of a quarter-half-quarter wavelength coating.}
  \label{fig:both quarter-quarter and quarter-half-quarter}
\end{figure}

In the quarter-quarter wavelength design, we employ two layers, each precisely a quarter of the desired wavelength in thickness. While the aim is to achieve minimal reflectance at a particular wavelength, the actual reflectance seldom reaches absolute zero due to the inherent limitations of material characteristics.

On the other hand, the quarter-half-quarter arrangement introduces an additional layer, half a wavelength thick, nestled between two quarter-wavelength films. This configuration is tailored to minimize reflectance across an extended spectrum of wavelengths.

Such multi-layered coatings surpass the effectiveness of single-layer films for anti-reflective purposes by accommodating a wider wavelength range. This versatility is particularly beneficial in applications where reducing reflection is paramount, such as in the manufacturing of optical lenses.

\subsection{Verification of Theoretical Results from Optics Literature}

This section aims to validate the principles outlined in \emph{Introduction to Optics} by Frank L. Pedrotti et al. through the simulation of multilayer film behavior, ensuring our computational results align with established optical theories.

Beginning with a straightforward approach, we examine the reflectance across various wavelengths for three distinct multilayer film configurations:
\begin{itemize}
    \item Films with quarter-quarter wavelength thickness.
    \item Films with quarter-half wavelength thickness.
    \item Films with quarter-half wavelength thickness, incorporating a different material for the half-wavelength layer.
\end{itemize}

\begin{figure}[ht!]
  \centering
  \includegraphics[width=0.4\textwidth]{Chapters/Figures/Chapter 4 Figures/Antireflecting Double Layer using Quarter and Half-Wavelength Thickness Films Layout.png}
  \caption{Antireflective double-layer arrangement. Source: \cite{pedrotti_introduction_2007}}
  \label{fig:antireflective_double_layer}
\end{figure}

We first examine films with quarter-quarter wavelength thickness. The reference from Pedrotti et al. specifies the lower thin dielectric layer as $\text{ZrO}_2$ and places $\text{CeF}_3$ directly above. The choice of materials is flexible, provided they adhere to the optimal ratio criterion for antireflectivity as defined in equation \ref{zero reflectance criterion - chap4}. This results in a refractive index ratio of $\frac{n_2}{n_1} = \frac{2.1}{1.65} \approx 1.273$, aligning closely with the ideal ratio of approximately 1.225 for normal incidence, given a substrate refractive index ($n_s$) of 1.5.

Expanding the spectrum of low reflectance within the visible region becomes achievable by diverging from the constraint of equal quarter-wavelength ($\lambda/4$) coatings. By integrating a layer of quarter-wavelength thickness as the second layer (considered from the bottom upwards), we attain more extensive zones of diminished reflectance. Consequently, in the scenario of item b, we employ magnesium fluoride ($\text{MgF}_2$), characterized by a refractive index of 1.38, as the quarter-wavelength thick material. The intermediate half-wavelength layer utilizes aluminum oxide, with a refractive index of 1.60. For item c, thorium dioxide, featuring a refractive index of 1.85, serves as the material for the half-wavelength thick layer, as noted in \cite{pedrotti_introduction_2007}.

For the specific wavelength of 550 nm, where the thicknesses of the quarter-wavelength ($\lambda/4$) and half-wavelength ($\lambda/2$) layers are calculated, the half-wavelength layer does not influence reflectance. In this case, the double-layer system acts akin to a solitary quarter-wavelength layer, resulting in a reflectance of 1.3\%. At wavelengths close to 550 nm, the presence of the half-wavelength layer contributes to maintaining reflectance levels below those achieved by a two quarter-wavelength layers. Below is the plot illustrating reflectance against wavelength for scenarios (a) through (c):

The graph below illustrates anti-reflectivity trends as detailed in the Optics textbook, indicating that while the reflectance at a wavelength of 550 nm stands at approximately 1.3\%, this value surpasses the performance of the quarter-quarter wavelength coating. Nevertheless, reflectance stays below this threshold across a wide wavelength span, extending from roughly 420 to 800 nm. This suggests that alternative configurations for dual-layer reflective films could be viable if the layers are not strictly constrained to quarter-wavelength multiples.

\begin{figure}[ht!]
  \centering
  \includegraphics[width=0.4\textwidth]{Chapters/Figures/Chapter 4 Figures/Antireflectivity Graphs in the Optics Book.png}
  \caption{Anti-reflectivity Graphs as shown in the Optics Textbook. Source: \cite{pedrotti_introduction_2007}}
  \label{fig:antireflectivity graphs in the Optics book}
\end{figure}

Presented below are the modeled results for scenarios (a), (b), and (c) as detailed previously, with corresponding legends found in figure \ref{fig:antireflectivity graphs in the Optics book}.

\begin{figure}[ht!]
  \centering
  \includegraphics[width=0.4\textwidth]{Chapters/Figures/Chapter 4 Figures/Antireflective Figure a.png}
  \caption{Modeling results of an anti-reflective coating with two layers, each a quarter-wavelength thick.}
  \label{fig:Antireflective Figure a}
\end{figure}

\begin{figure}[ht!]
  \centering
  \includegraphics[width=0.4\textwidth]{Chapters/Figures/Chapter 4 Figures/Antireflective Figure b.png}
  \caption{Reflectance outcomes for an anti-reflective coating with a bottom layer a quarter-wavelength thick and a top layer half-wavelength thick, using a material with a refractive index of 1.6.}
  \label{fig:Antireflective Figure b}
\end{figure}

\begin{figure}[ht!]
  \centering
  \includegraphics[width=0.4\textwidth]{Chapters/Figures/Chapter 4 Figures/Antireflective Figure c.png}
  \caption{Reflectance results for an anti-reflective double layer with a bottom layer a quarter-wavelength thick and a top layer half-wavelength thick, using a material with a refractive index of 1.85.}
  \label{fig:Antireflective Figure c}
\end{figure}

These results derive from the principles outlined in Chapter 2. Initially, the overall transfer matrix elements are calculated by multiplying the transfer matrices of the individual layers together. Within these calculations, the phase difference, $\delta$, varies with $\lambda$, aligning the film thickness with either $\lambda/4$ or $\lambda/2$ at the reference wavelength of 550 nm.

Subsequently, the reflection coefficient, as outlined in \ref{reflection coefficient in terms of transfer matrix terms}, is squared to produce reflectance as a wavelength function. COMSOL's Ray Optics module significantly simplifies this process by automating several steps, although careful consideration is needed for layer treatment, particularly regarding their classification as thin dielectric films.